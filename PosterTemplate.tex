\documentclass[landscape,paperheight=24in,fontscale=.45,paperwidth=36in,landscape,final]{baposter}

\usepackage{times}
\usepackage{calc}
\usepackage{amsmath}
\usepackage{amssymb}
\usepackage{relsize}
\usepackage{multirow}
\usepackage{bm}
\usepackage{tikz}
\usepackage{graphicx}
\usepackage{float}
\usepackage{multicol}
\usepackage{subfigure}
\usepackage{color}
\usepackage{pgfbaselayers}
\usepackage{subfigure}
\pgfdeclarelayer{background}
\pgfdeclarelayer{foreground}
\pgfsetlayers{background,main,foreground}

% \usepackage{helvet}
% %\usepackage{bookman}
% \usepackage{palatino}

\usepackage[T1]{fontenc}
\usepackage{ae}

% \newcommand{\captionfont}{\footnotesize}

\selectcolormodel{rgb}

\graphicspath{{images/}}

%%%%%%%%%%%%%%%%%%%%%%%%%%%%%%%%%%%%%%%%%%%%%%%%%%%%%%%%%%%%%%%%%%%%%%%%%%%%%%%%
% Multicol Settings
%%%%%%%%%%%%%%%%%%%%%%%%%%%%%%%%%%%%%%%%%%%%%%%%%%%%%%%%%%%%%%%%%%%%%%%%%%%%%%%%
\setlength{\columnsep}{0.7em}
\setlength{\columnseprule}{0mm}


%%%%%%%%%%%%%%%%%%%%%%%%%%%%%%%%%%%%%%%%%%%%%%%%%%%%%%%%%%%%%%%%%%%%%%%%%%%%%%%%
% Save space in lists. Use this after the opening of the list
%%%%%%%%%%%%%%%%%%%%%%%%%%%%%%%%%%%%%%%%%%%%%%%%%%%%%%%%%%%%%%%%%%%%%%%%%%%%%%%%
\newcommand{\compresslist}{%
\setlength{\itemsep}{1pt}%
\setlength{\parskip}{0pt}%
\setlength{\parsep}{0pt}%
}




\definecolor{silver}{cmyk}{0,0,0,0.3}
\definecolor{yellow}{cmyk}{0,0,0.9,0.0}
\definecolor{reddishyellow}{cmyk}{0,0.22,1.0,0.0}
\definecolor{black}{cmyk}{0,0,0.0,1.0}
\definecolor{darkYellow}{cmyk}{0,0,1.0,0.5}
\definecolor{darkSilver}{cmyk}{0,0,0,0.1}

\definecolor{lightyellow}{cmyk}{0,0,0.3,0.0}
\definecolor{lighteryellow}{cmyk}{0,0,0.1,0.0}
\definecolor{lighteryellow}{cmyk}{0,0,0.1,0.0}
\definecolor{lightestyellow}{cmyk}{0,0,0.05,0.0}

\definecolor{lightblues}{rgb}{.867, .918, .965}
\definecolor{mediumblues}{rgb}{.617, .789, .879}
\definecolor{darkblues}{rgb}{.191, .508, .738}

\definecolor{ColorBrew1}{rgb}{166, 97, 26}
\definecolor{ColorBrew2}{rgb}{223, 194, 125}
\definecolor{ColorBrew3}{rgb}{245, 245, 245}
\definecolor{ColorBrew4}{rgb}{128, 205, 193}
\definecolor{ColorBrew5}{rgb}{1, 133, 113}

\definecolor{textborder}{rgb}{0,0,156}
\definecolor{textheaderdark}{rgb}{0,0,156}
\definecolor{textheaderlight}{rgb}{0,0,156}

\definecolor{DukeBlue}{HTML}{001A57}
\definecolor{LBlue}{HTML}{336699}
\definecolor{Teal}{HTML}{018571}
%\definecolor{DukeBlue}{rgb}{0,0,.6}
%\definecolor{DukeBlue}{rgb}{0,0,.611764706}



%%%%%%%%%%%%%%%%%%%%%%%%%%%%%%%%%%%%%%%%%%%%%%%%%%%%%%%%%%%%%%%%%%%%%%%%%%%%%%
%%% Begin of Document
%%%%%%%%%%%%%%%%%%%%%%%%%%%%%%%%%%%%%%%%%%%%%%%%%%%%%%%%%%%%%%%%%%%%%%%%%%%%%%
\begin{document}

%%%%%%%%%%%%%%%%%%%%%%%%%%%%%%%%%%%%%%%%%%%%%%%%%%%%%%%%%%%%%%%%%%%%%%%%%%%%%%
%%% Here starts the poster
%%%---------------------------------------------------------------------------
%%% Format it to your taste with the options
%%%%%%%%%%%%%%%%%%%%%%%%%%%%%%%%%%%%%%%%%%%%%%%%%%%%%%%%%%%%%%%%%%%%%%%%%%%%%%
\typeout{Poster Starts}
\background{}

\begin{poster}{
  % Show grid to help with alignment
  grid=false,
  columns=4,
  % Column spacing
  colspacing=1em,
  % Color style
  bgColorOne=white, %specified the background color of the entire chart
  bgColorTwo=black, % not sure what this does
  borderColor=Teal,  % border color of boxes on poster
  headerColorOne=Teal,  % color of header
  headerColorTwo=black,  % not sure what this does
  headerFontColor=white, % color of font
  boxColorOne=white,
  boxColorTwo=white,
  % Format of textbox
  textborder=rounded,
  % Format of text header
  eyecatcher=true,
  headerborder=open,
  headerheight=0.08\textheight,
  headershape=roundedright,
  headershade=plain,
  headerfont=\large\textsc, %Sans Serif
  boxshade=plain,
%  background=shade-tb,
  background=plain,
  linewidth=2pt
  }
{
Eye Catcher, empty if option eyecatcher=no
}
{
\textcolor{Teal}{It's Who You Know}\\ \LARGE
\textcolor{Teal}{Replicating and Extending Philips (2012)}
}
{
\textcolor{LBlue}{Margaret J. Foster- mjf34@duke.edu}
}
  % University logo
  {{\begin{minipage}{13em}
    \hfill
    \includegraphics[height=5.5em]{logo_horizontal_300dpi.jpg}
  \end{minipage}}
  }


%%%%%%%%%%%%%%%%%%%%%%%%%%%%%%%%%%%%%%%%%%%%%%%%%%%%%%%%%%%%%%%%%%%%%%%%%%%%%%
\headerbox{Original Paper: Phillips (2012)}{name=introduction,column=0,row=0,span=1}
{
% %\section*{Source}
% %
\subsection*{Main Questions}
\begin{enumerate}
\item ...
\item ...
\end{enumerate}
\vspace{-2mm}
...
\vspace{-2mm}
\subsection*{Findings}
....
}
%
%
%%%%%%%%%%%%%%%%%%%%%%%%%%%%%%%%%%%%%%%%%%%%%%%%%%%%%%%%%%%%%%%%%%%%%%%%%%%%%%%%  
\headerbox{Alternate Model Specification}{name=data,column=0,below=introduction,span=1}
{
..
\vspace{-3mm}
\subsection*{Problem}
...
\vspace{-9mm}
\begin{center}
\begin{tabular}{l}
 \\ [2.0ex]
\includegraphics[height=68mm,width=68mm]{ROC.pdf}
\end{tabular}
\end{center}
\vspace{-11mm} 
\subsection*{Solution}
...
}


%%%%%%%%%%%%%%%%%%%%%%%%%%%%%%%%%%%%%%%%%%%%%%%%%%%%%%%%%%%%%%%%%%%%%%%%%%%%%%

\headerbox{Revised Point Estimates}
{name=results,column=1,span=1}
{
As this coefficient plot shows, introducing an alternative selection equation for MIDs does not greatly alter the point estimates for the religion/state dyad types for the outcome of interest, Severe Dispute.
\vspace{-9mm}
\begin{center}
\begin{tabular}{l}
 \\ [2.0ex]
\includegraphics[height=85mm,width=85mm]{CoefPlot.pdf}
\end{tabular}
\end{center}
\vspace{-4mm}
However, with fewer variables in the selection model, we can be \emph{more confident} in the robustness of the results of the outcome equation.
}

%%%%%%%%%%%%%%%%%%%%%%%%%%%%%%%%%%%%%%%%%%%%%%%%%%%%%%%%%%%%%%%%%%%%%%%%%%%%%%

%%%%%%%%%%%%%%%%%%%%%%%%%%%%%%%%%%%%%%%%%%%%%%%%%%%%%%%%%%%%%%%%%%%%%%%%%%%%%%
\headerbox{Predicted Probabilities by Dyad Type}
{name=dyads,column=2,span=2}
{
The plots below show the predicted probability of a severe dispute as time since a previous MID increases. Religion/state dyad types are plotted individually, holding all other variables at their mean, median, or at zero (dummies for other dyad types). 
\begin{center}
\begin{tabular}{lcc}
\includegraphics[height=93mm,width=165mm]{PredProbsLong.pdf}
\end{tabular}
\end{center}
\vspace{-3mm}
Dyads with religious and non-religious pairings (high ideological distance) start with higher probabilities of severity and do not decrease as much over time as ideologically-similar pairings. However, there is high uncertainty.
}

%%%%%%%%%%%%%%%%%%%%%%%%%%%%%%%%%%%%%%%%%%%%%%%%%%%%%%%%%%%%%%%%%%%%%%%%%%%%%%

\headerbox{Predictive Accuracy of the Model}
{name=sepplots,column=1,row=0,span=2,below=dyads}
{
In the separation plots below, predicted probabilities of severe disputes are ordered from low (far left) to high (far right), with actual severe disputes (1) represented in dark blue and non-events (0) represented in light brown. Ideally, all blue lines should be clustered on the right, \emph{separated} from non-events on the left. A trace line corresponds to the predicted probability values over observations. 
\vspace{-3mm}
\begin{center}
\includegraphics[height=40mm,width=140mm]{SepPlotAll.pdf}\\
\end{center}
\vspace{-12mm}
\begin{center}
\begin{tabular}{lcc}
\includegraphics[height=25mm,width=90mm]{SepPlot1.pdf}&\includegraphics[height=25mm,width=90mm]{SepPlot2.pdf}\\
\includegraphics[height=25mm,width=90mm]{SepPlot3.pdf}&\includegraphics[height=25mm,width=90mm]{SepPlot4.pdf}\\
\vspace{-10mm}
\end{tabular}
\end{center}
\vspace{-3mm}
The overall alternate model performs fairly well in-sample. A four-fold cross-validation reveals consistent out-of-sample performance. The plots also show, though, that the model tends to over-predict the outcome.
}


%%%%%%%%%%%%%%%%%%%%%%%%%%%%%%%%%%%%%%%%%%%%%%%%%%%%%%%%%%%%%%%%%%%%%%%%%%%%%%
\headerbox{Conclusions}
{name=conclusion,column=3,row=1,span=1,below=dyads}
{
Henne's work does pick up on an \textbf{apparent effect of religion-state relations on dispute severity}, but the finding is \textbf{overshadowed by issues of uncertainty.} Within the coefficient plot, a 95 percent confidence interval includes zero for five out of six religion/state dyad types. The predicted probabilities of severe disputes plotted for the different dyad types show that coefficients within a reasonable range could yield wildly different results. As evidenced by separation plots, the model can predict severe disputes relatively well, but it consistently overestimates.
\vspace{-2mm}
\subsection*{Bottom Line}
\vspace{-1mm}
Any theoretical or practical claims based on these results should be made with caution, but religion could be a viable variable of interest in international relations.
\vspace{-2mm}
\subsection*{Moving Forward}
\vspace{-2mm}
\begin{itemize}
\item Develop a theoretically robust and structurally-distinct selection equation to truly avoid correlated errors and misuse of selection methods.
\item Expand to other interstate interactions (trade, treaties, etc.) and religion's salience at different levels (beyond institutionalization, such as within public opinion, media, etc.) of national society.
\end{itemize}
}



\end{poster}
\end{document}